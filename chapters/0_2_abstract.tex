% \newgeometry{left=4.2cm,right=4.2cm}
% \pagestyle{abstractplain}
% \phantomsection
% \begin{center}
%     \sffamily \Large \textbf{\caps{EXECUTIVE SUMMARY}}
% \end{center}
% \markboth{Executive Summary}{}
% \addcontentsline{toc}{chapter}{Executive Summary}

% \vspace{0.2cm}
% The aim of this internship at the Princeton Plasma Physics Laboratory was to develop a transient numerical simulation of a laser-supported plasma (LSP), the core physical process powering laser-thermal pro\-pulsion (LTP), a concept that shows promise as a high thrust, high specific impulse propulsion system. This numerical solver was needed to support simulation work done at McGill University in Montreal, which provided inconclusive results on the temperature profile of a theoretical LTP thruster. Development of a transient simulation was believed to help troubleshoot the original simulation, along with potentially providing insight on several unsteady LSP phenomena. A secondary aim of this project was to lay the groundwork for constructing an experimental LTP facility to validate the results of the numerical simulations.

% A literature study on LSP experiment was first performed to gain a better understanding of the process, catalogue common experimental techniques used in the field, and identify any research gaps. Special attention was given to the types of lasers used, working gases, and ignition methods. Past facilities with significant research output were identified to eventually replicate their experimental methods.

% Development on the LSP solver then began with re-deriving existing governing equations for unsteady, axisymmetric flow, yielding a system of 3 partial-differential equations and an equation of state, similar to the Euler fluid equations, to be solved by the simulation. Conductive heat-transfer and gas-dynamics test problems were first solved numerically to practice the implementation of relevant finite-difference methods, namely FTCS and MacCormack al\-gorithms. This was followed by the implementation of the LSP solver itself in MATLAB, to make use of several components relating to laser absorption and thermodynamic properties already implemented and tested in a steady-state LSP solver, reducing development time. A modular code base inspired by the TU Delft Astrodynamics Toolbox design was selected to allow for future use and expansion of the code for further studies. Laser modelling proved to be a significant challenge as accurate physical models required complex implementation. The modular architecture of the simulator was found to be useful in integrating the various components of the code together.

% The experimental design of a new LTP facility benefitted greatly from the literature review, as critical optical elements and LSP ignition systems could be identified. Brainstorming of relevant research questions informed the selection of some diagnostics apparatus. This review of the current experimental state-of-the-art, along with the expertise of the PPPL on laser-plasma interaction, has provided confidence in the existence in research gaps and the feasibility of an experiment. Major progress was made on the implementation of the LSP simulation, with samples of the programming interface given in this report. Although issues with the chosen numerical algorithms continued to delay development past the end of the internship period, this project provided an excellent CFD learning experience, and the design of the code base will ensure a smooth continuation of the project in the future.

% Recommendations on further work for this project are given. The numerical solver is still to be completed with working numerical methods for the 1D and 2D case, along with physically realistic boundary conditions. Improvements to the solver would likely include a raytraced laser absorption model, and the decoupling of fluid dynamics physical model from the numerical solver. The design of an experimental LTP facility should move forward, informed by the literature study completed in this report, using a systems engineering approach to define experimental needs and requirements.

% \restoregeometry
% \pagestyle{fancy}

\begin{plainchp}{Executive Summary}
    \addcontentsline{toc}{chapter}{Executive Summary}

    The aim of this internship at the Princeton Plasma Physics Laboratory was to develop a transient numerical simulation of a laser-supported plasma (LSP), the core physical process powering laser-thermal pro\-pulsion (LTP), a concept that shows promise as a high thrust, high specific impulse propulsion system. This numerical solver was needed to support simulation work done at McGill University in Montreal, which provided inconclusive results on the temperature profile of a theoretical LTP thruster. Development of a transient simulation was believed to help troubleshoot the original simulation, along with potentially providing insight on several unsteady LSP phenomena. A secondary aim of this project was to lay the groundwork for constructing an experimental LTP facility to validate the results of the numerical simulations.

    A literature study on LSP experiment was first performed to gain a better understanding of the process, catalogue common experimental techniques used in the field, and identify any research gaps. Special attention was given to the types of lasers used, working gases, and ignition methods. Past facilities with significant research output were identified to eventually replicate their experimental methods.

    Development on the LSP solver then began with re-deriving existing governing equations for unsteady, axisymmetric flow, yielding a system of 3 partial-differential equations and an equation of state, similar to the Euler fluid equations, to be solved by the simulation. Conductive heat-transfer and gas-dynamics test problems were first solved numerically to practice the implementation of relevant finite-difference methods, namely FTCS and MacCormack al\-gorithms. This was followed by the implementation of the LSP solver itself in MATLAB, to make use of several components relating to laser absorption and thermodynamic properties already implemented and tested in a steady-state LSP solver, reducing development time. A modular code base inspired by the TU Delft Astrodynamics Toolbox design was selected to allow for future use and expansion of the code for further studies. Laser modelling proved to be a significant challenge as accurate physical models required complex implementation. The modular architecture of the simulator was found to be useful in integrating the various components of the code together.

    The experimental design of a new LTP facility benefitted greatly from the literature review, as critical optical elements and LSP ignition systems could be identified. Brainstorming of relevant research questions informed the selection of some diagnostics apparatus. This review of the current experimental state-of-the-art, along with the expertise of the PPPL on laser-plasma interaction, has provided confidence in the existence in research gaps and the feasibility of an experiment. Major progress was made on the implementation of the LSP simulation, with samples of the programming interface given in this report. Although issues with the chosen numerical algorithms continued to delay development past the end of the internship period, this project provided an excellent CFD learning experience, and the design of the code base will ensure a smooth continuation of the project in the future.

    Recommendations on further work for this project are given. The numerical solver is still to be completed with working numerical methods for the 1D and 2D case, along with physically realistic boundary conditions. Improvements to the solver would likely include a raytraced laser absorption model, and the decoupling of fluid dynamics physical model from the numerical solver. The design of an experimental LTP facility should move forward, informed by the literature study completed in this report, using a systems engineering approach to define experimental needs and requirements.

\end{plainchp}