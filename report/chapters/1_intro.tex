\chapter{Introduction}
% \addcontentsline{toc}{chapter}{Unnumbered Chapter}  % Uncomment to include in ToC
% \markboth{\MakeUppercase{Unnumbered Chapter}}{}
    The expansion of human activities beyond Earth's sphere of influence will demand the development of space propulsion systems providing high thrust and propellant efficiency. Such systems would facilitate rapid transit of crew and cargo across the solar system. In addition to the usual convenience and economic benefits, faster transit times would reduce astronaut exposure to the harsh radiation environment of interplanetary space, significantly mitigating the health risk factors of crewed missions.

    Laser-Thermal Propulsion (LTP) is a promising concept for such a propulsion system. Initially conceived in the 1970s by \textcite{kantrowitz_relevance_2015} among many other forms of Directed-Energy Propulsion (DEP), this concept consists of beaming laser power to a spacecraft, which then uses it to heat up propellant. Research into this concept has seen renewed interest in the last few years thanks to the development of low-cost, scalable, and modular fiber-laser technologies, and proposals by \textcite{lubin_roadmap_2022} to use such lasers for interstellar propulsion.

    Specifically, the work done in this thesis supports the efforts of research performed at McGill University on LTP. \textcite{duplay_design_2022} considered the application of LTP for a 45-day transit to Mars, showing that this concept could plausibly deliver a 1-ton payload to the planet for less than 1~\% of the propellant required by an equivalent mission powered by a chemical rocket engine.