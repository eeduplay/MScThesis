\chapter{Background}
    To best understand the work done in this thesis, some background information on laser-thermal propulsion is provided in this chapter. This is an abridged version of the literature review written before starting this thesis \cite{duplayReviewLaserThermalPropulsion2022}, which can be consulted for an in-depth study of past literature.

    \section{Working Principle}
        LTP is a directed-energy propulsion concept, a class of propulsion systems where energy is beamed to a spacecraft, usually as a laser. This energy is then used for propulsion in some shape or form. This allows the spacecraft to forego much of its power and propulsion system mass, increasing its propellant or payload mass budget. Some applications of DEP, such as lightsails, even bypass the rocket equation altogether, making them a promising avenue for interstellar missions, as shown by \textcite{lubinRoadmapInterstellarFlight2022}.

        ``Laser-thermal propulsion'' itself encompasses several concepts where the laser is used to energize a propellant stored aboard the spacecraft. \textcite{kantrowitzRelevanceSpace1971} first proposed this idea as way to reduce launch costs. Such concepts include pulsed concepts that ablate solid propellant or cause laser-supported detonations, as studied by \textcite{myraboPowerBeamingTechnologyLaser1984}, or laser heat-exchanger systems, as proposed by \textcite{kareLaserpoweredHeatExchanger1995}. The present study focuses on continuous-wave (CW) laser plasma propulsion, studied in detail by \textcite{keeferLaserSustainedPlasmas1989}: a continuous laser is used to sustain a laser-sustained plasma (LSP)\footnote{This physical phenomenon is also referred to as ``optical plasmotron'', ``light spark'', ``continuous optical discharge'', or ``laser-supported combustion wave'' in the literature.} core within a thrust chamber. This plasma absorbs laser energy and redistributes it to the propellant gas via conduction and radiation. 

    \section{Laser-Sustained Plasma}
    \section{Past Experiments}