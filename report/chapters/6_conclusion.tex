\chapter{Conclusion and further work}
    Already identified as an alternative to chemical propulsion in the 1970s, laser-thermal propulsion promises the high specific impulse and thrust necessary to unlock rapid transit in the solar system. Whether this promise can be achieved in practice and at scale is however yet to be seen. McGill University's Interstellar Flight Experimental Research Group hopes to revive practical research on LSP for propulsion applications, by attempting to replicate and move beyond some of the work done in the late 20\textsuperscript{th} century with the fiber-optic lasers considered for use in other directed-energy propulsion concepts such as interstellar lightsails and laser-electric propulsion. By contrast with more recent LSP research which focused on non-propulsion applications, this project aimed to study the thermal, heat deposition, and thrust characteristics of LSP.

    A brief review of LTP and LSP literature was provided. DEP and LTP concepts were discussed, including their advantages and drawbacks. Research on LSP was summarized, starting with the physics of inverse bremsstrahlung, the physical mechanism powering LSP, and moving on the models and observations made through an intense period of research between 1970 and 1990. Researchers at the time were considering the use of \ce{CO_2} lasers emitting \qty{10.6}{\um} radiation, and the implications of a switch to \qty{1.06}{\um} fiber-optic lasers were mentioned: while the range of these lasers is an order of magnitude greater, the lower IB absorption coefficient at this wavelength mandates higher laser power and/or pressure to sustain LSP. The design of past LSP facilities was briefly reviewed to provide context for the design choices made in creating such a facility at McGill, for the purposes of studying heat deposition and resulting thrust characteristics of Argon LSP.

    The design process of the LTP thruster laboratory model was then reported in detail. The system's design was driven in large part by the constraints set by the laser available for this experiment, capable of emitting \qty{3}{kW} pulses, but only for \qty{10}{ms}. The reasoning behind the decision to retrofit existing apparatus instead of opting for a clean-sheet design was discussed at length: given the many practical uncertainties surrounding the system, and the short timeline available for this project, the retrofit of a perhaps unoptimized test section was deemed preferable to inform the future design of a dedicated thruster model. This came at the cost of hindering thrust experiments, but this tradeoff paid off with the lessons learned in designing and rapidly testing an ignition system, developing diagnostic methodologies, and actually performing LSP experiments.

    Some brief modelling work was performed as part of this thesis, mainly to gain an understanding of the physics involved in inverse bremsstrahlung, and the relevant parameters driving laser-sustained plasma. Namely, chamber pressure is a key parameter of any thermal propulsion system, with high pressures providing greater thrust. In LTP, higher pressures are also beneficial to improve LSP absorption properties. Predicting peak absorption (with respect to temperature) is also helpful to estimate the maximum temperature reached in the LSP, as it has been found to be correlated.

    Finally, the first results of a series of experiments were reported. Preliminary ignition tests quickly revealed the challenges posed by spark-ignition. The use of such a system when constrained by a short laser pulse requires careful design to consistently align the laser focus and the spark. Successful LSP ignition using this system was difficult, but was achieved a few times, enough to build a small dataset on the laser absorption ability of LSP, which was observed to range from 70 to 90~\%. Initial estimates on the absorption coefficient, derived from the absorption data and high-speed footage, appear to agree with this study's modelling, though a more systematic absorption study would be needed to confirm this. For other experiments, wire-ignition was found to be far more reliable than spark-ignition, and allowed the replication of power threshold studies done in other LSP literature, finding that this ignition system provides a competitively low power threshold without the need of high precision optics. Spectral data acquired during these experiments should theoretically provide a measure of peak plasma temperature, but there appears to be methodology and/or processing issues to be resolved to provide a realistic temperature estimate. Flowing experiments were performed to explore the impact of incoming flow on LSP properties, but the feed system limitations only allowed for a cursory exploration.
    
    The recorded pressure change during and shortly after the LSP provided an insight into the heat deposition into the gas volume by the LSP. This ability will be crucial in a fully realized LTP system: to provide specific impulse on the order of \qty{3000}{s} yet high thrust, the LSP is meant to heat the surrounding propellant, and not be exhausted by itself (which would result in higher specific impulse but only for low thrust). The experimental data suggests a low heat deposition efficiency of around 15~\%, relative to the laser power incident on the LSP. Combined with the measured absorption, this builds an overall picture of the major loss factors involved in LSP: incomplete laser absorption and heat radiated to the walls or outside the test section appear to be responsible for 20~\% and 65~\% of the energy losses, respectively. This provides a baseline on efficiency that can now be improved on with a variety of strategies suggested in LTP literature. The peculiar shape of the pressure profile, with its local maximum and minimum, should be the subject of further study.

    The objectives set for this project, to build an Argon LTP thruster model, may not have been entirely met. Issues encountered with the unoptimized test section and its impact on thrust measurement meant that meaningful thrust experiments could not be performed. However, a method to determine heat deposition into the working gas was developed based on the the pressure change of the test section, providing a baseline on heat deposition efficiency, which can be used to design the next iteration of an LTP thruster at McGill University. In this regard, the project is successful in initiating a new experimental research effort on LTP at the IFERG, and the questions and issues raised across various aspects of this project could motivate several new, more targeted studies.

    \section{Further work}
        As this thesis project's \emph{raison d'être} was to lay the groundwork for experimental research on LSP and LTP at McGill, there are many opportunities for further work. A selection of such opportunities is given below.

        \paragraph{Optimization of the test section} Although the retrofit of the cavitation experiment's test section enabled rapid experimentation, its non-optimal design posed several challenges, some of which were already discussed in \autoref{sec:design_testSection}. The test section mass was particularly problematic for thrust experiments. Further research on LSP and LTP will be limited without the development of a new LSP generator or prototype thruster optimized for this project. Such optimizations would include:
        \begin{itemize}
            \item Opting for a lighter material for the pressure vessel, likely aluminum, to minimize weight
            \item Reducing the overall length and diameter of the vessel. This would both provide more flexibility in terms of beam geometry, allowing the use of shorter lenses or placing the laser focus at different locations in the chamber, to potentially optimize the LSP location relative to the nozzle. Smaller dimensions would also reduce the overall weight.
            \item Smaller observation windows. Although the current side windows offer excellent visibility throughout the length of the test section, their slender geometry and length mandated the use of cumbersome steel mounting clamps. These clamps added considerable mass to the test section. Opting for smaller round windows bolted directly into the pressure vessel's body using aluminum mounts may be sufficient.
        \end{itemize}
        Such improvements would greatly facilitate the development of an appropriate thrust stand and provide greater beam-shaping flexibility without compromising on laser absorption measurements.

        \paragraph{Improved spark ignition system} As discussed in \autoref{sec:ignitiontest}, the original spark igniter design for this study proved difficult to work with, as the large spark gap and side-by-side electrodes created inconsistent arc paths that would rarely intersect with the laser beam path. Although good results were obtained with wire-ignition, spark-ignition is still thought to be optimal for future experiments, as its advantages over wire-ignition would be worth the additional development efforts. As a reminder, they are as follows:
        \begin{itemize}
            \item Uninterrupted laser beam path allows the measurement of absorbed laser power and determining LSP absorption coefficient. It also does not impede the downstream growth of the LSP.
            \item Sparks can be generated at will without consuming material between each test. Several experiments could potentially be done in quick succession without re-aligning optics or replacing ignition wire.
        \end{itemize}
        In order to improve spark consistency and ignition reliability, several improvements could be made to both the spark plug design and how it integrates in the test section:
        \begin{itemize}
            \item The electrodes should be in a co-axial configuration, as was done for several LSP experiments in the literature (\textcite{luCharacteristicDiagnosticsLaserStabilized2022, zimakovInteractionNearIRLaser2016,matsuiGeneratingConditionsArgon2019}). This may improve the consistency of the arc path, and would enable precise mechanical control of electrode distance more easily than with side-by-side electrodes.
            \item To accommodate for such an electrode arrangement, the test section should be modified with instrumentation ports along the opposite wall of the cylinder. Each port should ideally be precisely matched with another port facing it.
            \item Electrode tip distance should be reduced down to about a millimeter to favor arcing even at \qty{20}{bar}, and to constraint possible arc paths to those intersecting with the laser focus. Ideally, this gap should be adjustable in order to adapt the electrode distance based on the test pressure.
            \item Discussions with researchers experienced with spark igniters suggested that a sharp tipped electrode paired with a rounded electrode gave better results.
        \end{itemize}

        \paragraph{Specific impulse measurement} One of the ultimate goals of the LTP project at McGill University is to demonstrate the feasibility of the concept, and show that a specific impulse of \qtyrange{1000}{3000}{s} is possible under the right conditions. While the roadmap to this sort of performance is long and would involve a switch to Hydrogen as a working gas, the experiment should be set-up such that mass-flow rates can be controlled and/or measured, enabling the calculation of exhaust velocity when combined with thrust data. This can be done using a mass flowmeter, or by controlling mass flow using different diameters of choked orifices, and operating the facility in a double-choked configuration (choked at inlet and exhaust nozzle).

        \paragraph{Absorption measurements in flowing conditions} The nozzles used in this study could be easily fabricated and swapped on the test section, but made it impossible to acquire an accurate measure of the laser power transmitted through the LSP and out of the test section, as the orifice size was significantly smaller than the laser beam. Designing a nozzle module that allows such a measurement would permit the study of the effect of flow on laser absorption. Poor laser absorption is one of the main efficiency loss mechanisms for an LTP thruster, so being able to measure it in flowing conditions would be valuable. This can be done either by using a regular laser window mount with an off-axis nozzle, or by designing an annular nozzle around the laser window (whether this option is worth the considerable design effort is debatable).

        \paragraph{Additional spectrometry and thermal imaging} As discussed in \autoref{sec:results_spectroscopy}, there is much room to improve this experiment's spectroscopy methodologies. The spectrometer's fiber termination should be equipped with a collimator to sample precise points in the LSP, which should improve the spectral data for temperature estimation with the Boltzmann plot method. Once this is corrected, the collimator could be mounted on opto-mechanical stages to precisely position it relative to the LSP, enabling the construction of temperature maps of the LSP, which could be compared to axi-symmetric numerical models currently in development at McGill (\textcite{baoTwoDimensionalSimulationLaser}). In addition to spectroscopy, infrared thermal imaging could potentially be used to study the change in temperature of the cooler surrounding gas, providing additional data on the effective heat deposition from the LSP.