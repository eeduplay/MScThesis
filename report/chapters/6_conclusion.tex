\chapter{Conclusion and further work}

    \section{Further work}
        As this thesis project's \emph{raison d'être} was to lay the groundwork for experimental research on LSP and LTP at McGill, there are many opportunities for further work. A selection of such opportunities is given below.

        \paragraph{Optimization of the test section} Although the retrofit of the cavitation experiment's test section enabled rapid experimentation, its non-optimal design posed several challenges, some of which were already discussed in \autoref{sec:design_testSection}. The test section mass was particularly problematic for thrust experiments. Further research on LSP and LTP will be limited without the development of a new LSP generator or prototype thruster optimized for this project. Such optimizations would include:
        \begin{itemize}
            \item Opting for a lighter material for the pressure vessel, likely aluminum, to minimize weight
            \item Reducing the overall length and diameter of the vessel. This would both provide more flexibility in terms of beam geometry, allowing the use of shorter lenses or placing the laser focus at different locations in the chamber, to potentially optimize the LSP location relative to the nozzle. Smaller dimensions would also reduce the overall weight.
            \item Smaller observation windows. Although the current side windows offer excellent visibility throughout the length of the test section, their slender geometry and length mandated the use of cumbersome steel mounting clamps. These clamps added considerable mass to the test section. Opting for smaller round windows bolted directly into the pressure vessel's body using aluminum mounts may be sufficient.
        \end{itemize}
        Such improvements would greatly facilitate the development of an appropriate thrust stand and provide greater beam-shaping flexibility without compromising on laser absorption measurements.

        \paragraph{Improved spark ignition system} As discussed in \autoref{sec:ignitiontest}, the original spark igniter design for this study proved difficult to work with, as the large spark gap and side-by-side electrodes created inconsistent arc paths that would rarely intersect with the laser beam path. Although good results were obtained with wire-ignition, spark-ignition is still thought to be optimal for future experiments, as its advantages over wire-ignition would be worth the additional development efforts. As a reminder, they are as follows:
        \begin{itemize}
            \item Uninterrupted laser beam path allows the measurement of absorbed laser power and determining LSP absorption coefficient. It also does not impede the downstream growth of the LSP.
            \item Sparks can be generated at will without consuming material between each test. Several experiments could potentially be done in quick succession without re-aligning optics or replacing ignition wire.
        \end{itemize}
        In order to improve spark consistency and ignition reliability, several improvements could be made to both the spark plug design and how it integrates in the test section:
        \begin{itemize}
            \item The electrodes should be in a co-axial configuration, as was done for several LSP experiments in the literature (\textcite{luCharacteristicDiagnosticsLaserStabilized2022, zimakovInteractionNearIRLaser2016,matsuiGeneratingConditionsArgon2019}). This may improve the consistency of the arc path, and would enable precise mechanical control of electrode distance more easily than with side-by-side electrodes.
            \item To accommodate for such an electrode arrangement, the test section should be modified with instrumentation ports along the opposite wall of the cylinder. Each port should ideally be precisely matched with another port facing it.
            \item Electrode tip distance should be reduced down to about a millimeter to favor arcing even at \qty{20}{bar}, and to constraint possible arc paths to those intersecting with the laser focus. Ideally, this gap should be adjustable in order to adapt the electrode distance based on the test pressure.
            \item Discussions with researchers experienced with spark igniters suggested that a sharp tipped electrode paired with a rounded electrode gave better results.
        \end{itemize}

        \paragraph{Specific impulse measurement} One of the ultimate goals of the LTP project at McGill University is to demonstrate the feasibility of the concept, and show that a specific impulse of \qtyrange{1000}{3000}{s} is possible under the right conditions. While the roadmap to this sort of performance is long and would involve a switch to Hydrogen as a working gas, the experiment should be set-up such that mass-flow rates can be controlled and/or measured, enabling the calculation of exhaust velocity when combined with thrust data. This can be done using a mass flowmeter, or by controlling mass flow using different diameters of choked orifices, and operating the facility in a double-choked configuration (choked at inlet and exhaust nozzle).

        \paragraph{Absorption measurements in flowing conditions} The nozzles used in this study could be easily fabricated and swapped on the test section, but made it impossible to acquire an accurate measure of the laser power transmitted through the LSP and out of the test section, as the orifice size was significantly smaller than the laser beam. Designing a nozzle module that allows such a measurement would permit the study of the effect of flow on laser absorption. Poor laser absorption is one of the main efficiency loss mechanisms for an LTP thruster, so being able to measure it in flowing conditions would be valuable. This can be done either by using a regular laser window mount with an off-axis nozzle, or by designing an annular nozzle around the laser window (whether this option is worth the considerable design effort is debatable).

        \paragraph{Additional spectrometry and thermal imaging} As discussed in \autoref{sec:results_spectroscopy}, there is much room to improve this experiment's spectroscopy methodologies. The spectrometer's fiber termination should be equipped with a collimator to sample precise points in the LSP, which should improve the spectral data for temperature estimation with the Boltzmann plot method. Once this is corrected, the collimator could be mounted on opto-mechanical stages to precisely position it relative to the LSP, enabling the construction of temperature maps of the LSP, which could be compared to axi-symmetric numerical models currently in development at McGill (\textcite{baoTwoDimensionalSimulationLaser}). In addition to spectroscopy, infrared thermal imaging could potentially be used to study the change in temperature of the cooler surrounding gas, providing additional data on the effective heat deposition from the LSP.